\documentclass[12pt]{article}
\usepackage{graphicx}

\usepackage{kotex}
\usepackage{amsmath}
\usepackage{amsfonts}
\usepackage{amssymb}
\usepackage{mathtools}
\usepackage{geometry}
\usepackage{amsthm}
\usepackage{verbatim}
\usepackage{pgfplots}
\geometry{
	top = 30mm,
	left = 25mm,
	right = 25mm,
	bottom = 30mm
}
\geometry{a4paper}

\renewcommand{\baselinestretch}{1.4}


\theoremstyle{definition}
\newtheorem{theorem}{\sffamily 정리}[section]
\theoremstyle{definition}
\newtheorem{definition}[theorem]{\sffamily 정의}  
\newtheorem{problem}[theorem]{\sffamily 연습문제}

\newcommand{\defn}[1]{\begin{definition}#1\end{definition}~}
\newcommand{\thm}[1]{\begin{theorem}#1\end{theorem}~}
\newcommand{\prob}[1]{\begin{problem}#1\end{problem}~}

\usepackage{titlesec}

\pgfmathdeclarefunction{gauss}{2}{%
	\pgfmathparse{1/(#2*sqrt(2*pi))*exp(-((x-#1)^2)/(2*#2^2))}%
}

\title{\textbf{Introductory Statistics}}
\author{Sungchan Yi}
\date{January 2019}

\titleformat*{\section}{\large\bfseries}

\newcommand{\p}[2]{_{#1}\text{P}_{#2}}
\newcommand{\ds}{\displaystyle}
\newcommand{\pr}{\text{P}}
\newcommand{\cndpr}[2]{\pr\!\left(#1\,|\,#2\right)}
\newcommand{\indep}{\mathrel{\!\perp \!\!\!\perp\!}}
\newcommand{\ex}{\mathbf{E}}
\newcommand{\var}{\mathbf{V}}
\newcommand{\bi}{\mathbf{B}}
\newcommand{\mc}[1]{\mathcal{#1}}
\newcommand{\mr}[1]{\mathrm{#1}}
\renewcommand{\qed}{\hfill\ensuremath{\square}}


\begin{document}
\maketitle

\tableofcontents
\pagebreak

\section{자료의 생성}
\textbf{통계학}(statistics)이란, 주어진 문제에 대하여 합리적인 답을 줄 수 있도록 숫자로 표시되는 정보를 \textbf{수집}하고 정리하며, 이를 해석하고 \textbf{신뢰성 있는 결론}을 이끌어 내는 방법을 연구하는 과학의 한 분야이다.\\\\
그러면 두 가지 질문이 생긴다.
\begin{enumerate}
	\item \textbf{수집}: 어떻게 수집해야 전체를 잘 대표할 수 있는가?
	\item \textbf{신뢰성 있는 결론}: 어떻게 신뢰성을 측정하여 결론을 내릴 것인가?
\end{enumerate}
\vspace{3mm}
\defn{~
\begin{itemize}
	\item \textbf{추출단위}(sampling unit): 전체를 구성하는 각 개체
	\item \textbf{특성값}(characteristic): 각 추출단위의 특성을 나타내는 값
	\item \textbf{모집단}(population): 관심의 대상이 되는 모든 추출단위의 특성값을 모아 놓은 것\\ 추출단위의 개수가 유한하면 \textbf{유한모집단}, 무한하면 \textbf{무한모집단}이라 한다.
	\item \textbf{표본}(sample): 실제로 관측한 추출단위의 특성값의 모임
\end{itemize}
}

\defn{(자료의 종류)
	\begin{enumerate}
		\item \textbf{범주형 자료}(categorical data), \textbf{질적 자료}(qualitative data)는 관측 결과가 몇 개의 범주 또는 항목의 형태로 나타나는 자료이다.
		\begin{itemize}
			\item \textbf{명목자료}(nominal data): 순위의 개념이 없다. 예) 혈액형, 성별
			\item \textbf{순서자료}(ordinal data): 순위의 개념을 갖는다. 예) A $\sim$ F 학점, 9등급제
		\end{itemize}
		\item \textbf{수치형 자료}(numerical data), \textbf{양적 자료}(quantitative data)는 자료 자체가 숫자로 표현되며 숫자 자체가 자료의 속성을 반영한다.
		\begin{itemize}
			\item \textbf{이산형 자료}(discrete data) 예) 교통사고 건수
			\item \textbf{연속형 자료}(continuous data) 예) 키, 몸무게
		\end{itemize}
	\end{enumerate}}

\defn{(통계학의 분류)
\begin{itemize}
	\item \textbf{기술통계학}(descriptive statistics)은 표나 그림 또는 대표값 등을 통하여 수집된 자료의 특성을 쉽게 파악할 수 있도록 자료를 정리$\cdot$요약 하는 방법을 다루는 분야이다.
	\item \textbf{추측통계학}(inferential statistics)은 표본에 내포된 정보를 분석하여 모집단의 여러가지 특성에 대하여 과학적으로 추론하는 방법을 다루는 분야이다.	
\end{itemize}
}

\defn{$N$개의 추출단위로 구성된 유한모집단에서 $n$개의 추출단위를 비복원추출할 때, $_N \mathrm{C}_n$개의 모든 가능한 표본들이 동일한 확률로 추출되는 방법을 \textbf{단순랜덤추출법}(simple random sampling)이라 하고, 이 방법을 위해서는 난수표(random number table)나 난수생성기(random number generator) 등을 이용한다. 그리고 단순랜덤추출로 얻은 표본을 \textbf{단순랜덤표본}(simple random sample)이라 한다. 
}

\defn{(통계적 실험)
\begin{itemize}
	\item 실험이 행해지는 개체를 \textbf{실험단위}(experimental unit/subject)라 하고, 각각의 실험단위에 특정한 실험환경 또는 실험조건을 가하는 것을 \textbf{처리}(treatment)라 한다.
	\item 처리를 받는 집단을 \textbf{처리집단}(treatment group), 처리를 받지 않은 집단을 \textbf{대조집단}(control group)이라 한다.
	\item 실험환경이나 실험조건을 나타내는 변수를 \textbf{인자}(factor)라 하고, 인자가 취하는 값을 그 인자의 \textbf{수준}(level)이라 한다. 
	\item 인자에 대한 반응을 나타내는 변수를 \textbf{반응변수}(response variable)라 한다.
	\item 실험단위가 처리집단이나 대조집단에 들어갈 기회를 동등하게 부여하는 방법을 \textbf{랜덤화}(randomization)라 한다.
	\item 랜덤화에 의해 모든 실험단위를 각 처리에 배정하는 실험계획을 \textbf{완전 랜덤화 계획}(completely randomized design)이라 한다.
	\item 실험 이전에 동일 처리에 대한 반응이 유사할 것으로 예상되는 실험단위들끼리 모은 것을 \textbf{블록}(block)이라 하고, 랜덤화에 의해 모든 블록을 각 처리에 배정하는 실험계획을 \textbf{블록화}(randomized block design)라 한다.
\end{itemize}
}

\defn{\textbf{(통계적 실험계획의 원칙)}
\begin{enumerate}
	\item \textbf{대조}(control): 관심 인자 이외의 다른 외부 인자의 효과를 극소화하고, 처리에 대한 대조집단을 통해 비교 실험을 한다.
	\item \textbf{랜덤화}(randomization): 완전랜덤화계획
	\item \textbf{반복 시행}(replication): 처리효과의 탐지를 용이하게 하기 위해 반복 시행한다. 
\end{enumerate}
}


\pagebreak

\section{대표값과 산포도}

\pagebreak
\section{순열과 조합}
\defn{$0!=1$, $n! = \displaystyle \prod_{i=1}^n i  = n\cdot(n-1)\cdot\cdots\cdot2\cdot 1 \: (n\geq 1)$ 로 정의하고, $!$ 는 \textbf{팩토리얼}(factorial)이라 읽는다.}
\defn{서로 다른 $n$개의 원소에서 서로 다른 $r$개를 택하여 일렬로 배열하는 것을 $n$개에서 $r$개를 택하는 \textbf{순열}(permutation)이라 하고, 기호로 $\p{n}{r}$ 와 같이 나타낸다.}

\thm{$\p{n}{r} = n(n-1)\cdots(n-r+1) = \dfrac{n!}{(n-r)!}$ \quad (단, $0\leq r\leq n$)}

\defn{서로 다른 $n$개의 원소에서 순서를 생각하지 않고 $r$개를 택하는 것을 $n$개에서 $r$개를 택하는 \textbf{조합}(combination)이라 하고, 기호로 $_{n}\text{C}_{r}$ 또는 $\displaystyle {n \choose r}$ 과 같이 나타낸다.}

\thm{$\displaystyle {n \choose r} = \dfrac{\p{n}{r}}{r!} = \dfrac{n!}{r!(n-r)!}$ \quad (단, $0\leq r\leq n$)}

\thm{(조합의 성질) 
\begin{enumerate}
	\item[(1)] $\displaystyle {n \choose r} = {n \choose n - r}$ \quad (단, $0\leq r\leq n$) (대칭성) 
	\item[(2)] $\ds {n\choose r} = {n-1\choose r} + {n-1\choose r-1} $ \quad (단, $1\leq r\leq n-1$) \textbf{(파스칼 법칙)} 
\end{enumerate}}

\defn{서로 다른 $n$개의 원소에서 중복을 허락하여 $r$개를 택하는 순열을 $n$개에서 $r$개를 택하는 \textbf{중복순열}이라 하고, 기호로 $_n \Pi_r$ 과 같이 나타낸다.}

\defn{서로 다른 $n$개의 원소에서 중복을 허락하여 $r$개를 택하는 조합을 $n$개에서 $r$개를 택하는 \textbf{중복조합}이라 하고, 기호로 $_n \text{H}_r$ 과 같이 나타낸다.}

\thm{$\ds _n \Pi_r = n^r, \quad _n \text{H}_r = {n+r-1 \choose r}$.}

\thm{$n\in\mathbb{N}$ 에 대하여, $$(x+y)^n = \sum_{r=0}^n {n\choose r} x^{n-r}y^{r} = {n \choose 0}x^ny^0 + {n\choose 1}x^{n-1}y^1+\cdots + {n\choose n}x^0y^n$$ 이다. 이를 $ (a+b)^n $에 대한 \textbf{이항정리}(binomial theorem)라 하고, $\ds{n\choose r}x^{n-r}y^r$을 전개식의 \textbf{일반항}, 전개식의 각 항의 계수 $\ds {n\choose r} $들을 \textbf{이항계수}라 한다.}

\thm{(이항계수의 성질)
\begin{enumerate}
	\item[(1)] $\ds (1+x)^n = \sum_{r=0}^n {n\choose r}x^r={n \choose 0} + {n\choose 1}x + \cdots + {n\choose n}x^n$ \quad (for all $x\in \mathbb{C}$)
	\item[(2)] $\ds \sum_{r=0}^n {n\choose r} = {n \choose 0} + {n \choose 1}+\cdots + {n \choose n} = 2^n$
	\item[(3)] $\ds \sum_{r=0}^n (-1)^r {n\choose r} = {n \choose 0} - {n \choose 1} + {n \choose 2} - {n \choose 3} +\cdots + (-1)^n{n \choose n} = 0$
	\item[(4)] $\ds \sum_{r=0}^nr{n\choose r} ={n \choose 1} + 2\cdot {n \choose 2} + \cdots + n\cdot {n \choose n} = n\cdot 2^{n-1}$
	\item[(5)] $\ds \sum_{r=0}^nr^2{n\choose r} = 2^2\cdot {n \choose 2} +3^2\cdot {n \choose 3} + \cdots + n^2\cdot {n \choose n} = n(n+1)\cdot 2^{n-2}$
	\item[(6)] $\ds \sum_{r=0}^n \frac{1}{r+1}{n\choose r}=\frac{1}{1}{n\choose 0} + \frac{1}{2}{n\choose 1} + \cdots + \frac{1}{n+1}{n\choose n} = \frac{1}{n+1}\left(2^{n+1}-1\right)$
\end{enumerate}}

\thm{$n\in\mathbb{N}$ 에 대하여, $$(x_1+x_2+\cdots+x_m)^n = \sum_{r_1+r_2+\cdots+r_m=n} {n\choose r_1, r_2, \dots, r_m}x_1^{r_1}x_2^{r_2}\cdots x_m^{r_m} $$ 이고, 이를 $(x_1+x_2+\cdots+x_m)^n$에 대한 \textbf{다항정리}(multinomial theorem)라 한다. 이 때 $\ds{n\choose r_1, r_2, \dots, r_m}$를 \textbf{다항계수}라 하고, 다음과 같이 정의한다. $${n\choose r_1, r_2, \dots, r_m} = \frac{n!}{r_1!\cdot r_2!\cdot \cdots \cdot r_m!}$$}

\defn{서로 다른 $n$개의 원소를 원형으로 배열하는 순열을 \textbf{원순열}이라 하고, 그 경우의 수는 $(n-1)!$ 이다.}

\thm{(원순열의 일반공식) $n$개 중에서 서로 같은 것이 $p_1, p_2, \dots, p_k$개씩 있을 때, 이 $n\,(=p_1+\cdots+p_k)$개를 원형으로 배열하는 방법(원순열)의 수는 다음과 같다.
$$\frac{1}{n}\sum_{d\,|\,g} \left\{\phi(d) {\frac{n}{d}\choose \frac{p_1}{n},\frac{p_2}{n},\dots,\frac{p_k}{n}}  \right\}$$
단, $g = \gcd(p_1, \dots, p_k)$, $d>0$ 이고 $\phi(d)$는 $d$ 이하의 자연수 중에서 $d$ 와 서로소인 자연수의 개수로 정의된다. 
}
\pagebreak
\section{확률의 뜻과 활용}
확률은 모집단에서 표본을 추출할 때, 특정 성질을 만족하는 표본이 관측될 가능성에 대한 측도로, 표본을 바탕으로 \textbf{모집단에 대한 결론을 이끌어낼 때 논리적 근거}가 된다.
\defn{같은 조건 아래에서 반복할 수 있고, 그 결과가 우연에 의하여 결정되는 실험이나 관찰을 \textbf{시행}이라고 한다. 어떤 시행에서 일어날 수 있는 모든 가능한 결과 전체의 집합을 \textbf{표본공간}(sample space $S$)이라 하고, 표본공간의 부분집합을 \textbf{사건}(event)이라고 한다.}

\defn{표본공간의 부분집합 중에서 원소의 개수가 한 개인 집합을 \textbf{근원사건}이라 하고, 반드시 일어나는 사건은 \textbf{전사건}, 절대로 일어나지 않는 사건은 \textbf{공사건}($\varnothing$)이라 한다.}


\defn{두 사건 $A, B$에 대하여, $A$ 또는 $B$가 일어나는 사건을 $A$와 $B$의 \textbf{합사건}이라 하고, $A\cup B$ 로 나타낸다. 그리고 $A$와 $B$가 동시에 일어나는 사건을 $A$와 $B$의 \textbf{곱사건}이라 하고, $A\cap B$ 로 나타낸다.}

\defn{표본공간 $S$의 부분집합인 두 사건 $A, B$에 대하여 $A\cap B = \varnothing$ 이면 $A$와 $B$는 서로 \textbf{배반사건}(disjoint)이라 한다. 또, 사건 $A$가 일어나지 않는 사건을 사건 $A$의 \textbf{여사건}이라 하고, $A^C$로 나타낸다.}

\defn{표본공간 $S$의 공사건이 아닌 사건 $A_1, \dots, A_n$이 다음 조건을 만족하면,
\begin{enumerate}
	\item[(1)] $\ds \bigcup_{i=1}^n A_i = S$
	\item[(2)] $A_i\cap A_j = \varnothing\:$  $\big(\text{for all }1\leq i \neq j \leq n\big)$ \quad (pairwise disjoint)
\end{enumerate} 사건 $A_1, \dots, A_n$을 $S$의 \textbf{분할}(partition)이라 한다.
}

\defn{어떤 시행에서 사건 $A$가 일어날 가능성을 수로 나타낸 것을 사건 $A$가 일어날 \textbf{확률}이라 하고, 기호로 $\pr(A)$와 같이 나타낸다.}

\defn{(수학적 확률) 어떤 시행의 표본공간 $S$가 $m$개의 근원사건으로 이루어져 있고, \textbf{각 근원사건이 일어날 가능성이 모두 같은 정도로 기대될 때}, 사건 $A$가 $r$개의 근원사건으로 이루어져 있으면 사건 $A$가 일어날 확률은 다음과 같다.$$\pr(A) = \frac{\text{(사건 }A\text{가 일어나는 경우의 수)}}{\text{(모든 경우의 수)}} =\frac{n(A)}{n(S)} = \frac{r}{m}$$}

\defn{(통계적 확률) 같은 시행을 $n$번 반복하여 사건 $A$가 일어난 횟수를 $r_n$이라고 하자. 이 때, 시행 횟수 $n$이 한없이 커짐에 따라 그 상대도수 $r_n/n$ 은 $\pr(A)$에 가까워진다. $$\pr(A)=\lim_{n\rightarrow \infty} \frac{r_n}{n}$$} 

\defn{(기하학적 확률) 연속적인 변량을 크기로 갖는 표본공간의 영역 $S$ 안에서 각각의 점을 잡을 가능성이 같은 정도로 기대될 때, 영역 $S$에 포함되어 있는 영역 $A$에 대하여 영역 $S$에서 임의로 잡은 점이 영역 $A$에 속할 확률은 다음과 같다. $$\pr(A) = \frac{\text{(영역 }A\text{의 크기)}}{\text{(영역 }S\text{의 크기)}}$$ }

\defn{(확률의 공리 - Axioms of Probability) 표본공간 $S$와 사건 $A$에 대하여,
\begin{enumerate}
	\item[(1)] $0\leq \pr(A)\leq 1$
	\item[(2)] $\pr(S) = 1$ \vspace{-1.5mm}
	\item[(3)] 서로 배반인 사건열 $A_1, A_2, \dots$ 에 대해 $\ds \pr\left(\bigcup_{i=1}^\infty A_i\right) = \sum_{i=1}^\infty \pr(A_i)$
\end{enumerate}
}

\thm{\textbf{(확률의 기본 성질)} 사건 $A, B$에 대하여 다음이 성립한다.
\begin{enumerate}
	\item[(1)] $\pr(\varnothing) = 0$
	\item[(2)] $\pr(A\cup B) = \pr(A) + \pr(B) - \pr(A\cap B)$ \quad \textbf{(확률의 덧셈정리)}
	\item[(3)] $\pr(A^C) = 1-\pr(A)$ \quad (여사건의 확률)

\end{enumerate}
}

\thm{사건 $A, B, C$에 대하여 다음이 성립한다. $$\pr(A\cup B\cup C) = \pr(A) + \pr(B) +\pr(C) - \pr(A\cap B)-\pr(B\cap C) - \pr(C\cap A) + \pr(A\cap B\cap C)$$}
\thm{\textbf{(포함 배제 원리)} 사건 $A_1, \dots, A_n$에 대하여 다음이 성립한다.
$$\pr\left(\bigcup_{i=1}^n A_i\right) = \sum_{k=1}^n (-1)^{k+1} \left(\sum_{1\leq i_1<\cdots< i_k\leq n} \pr\left(A_{i_1}\cap \cdots \cap A_{i_k}\right)\right)$$
}

\pagebreak

\section{조건부확률}
\defn{확률이 $0$이 아닌 두 사건 $A, B$에 대하여 사건 $A$가 일어났을 때, 사건 $B$가 일어날 확률을 사건 $A$가 일어났을 때의 사건 $B$의 \textbf{조건부확률}(conditional probability)이라 하고, 기호로 $\pr(B\,|\,A)$ 와 같이 나타낸다. 이는 다음과 같이 계산한다.$$\pr(B\,|\,A) = \frac{\pr(A\cap B)}{\pr(A)} \quad \big(\pr(A)>0\big)$$}

\thm{\textbf{(확률의 곱셈정리)} 공사건이 아닌 두 사건 $A, B$에 대하여 다음이 성립한다.$$\pr(A\cap B) = \pr(B\,|\,A)\pr(A) = \pr(A\,|\,B)\pr(B)$$}

\defn{두 사건 $A, B$에 대하여 사건 $A$가 일어났을 때의 사건 $B$의 조건부확률이 사건 $B$가 일어날 확률과 같을 때, 즉 $$\pr(B\,|\,A) = \pr(B\,|\,A^C) = \pr(B)$$ 이면, 두 사건 $A, B$는 서로 \textbf{독립}(independent)이라 하고, 기호로 $A\indep B$ 와 같이 나타낸다. 두 사건이 독립이 아닐 때는 \textbf{종속}이라 한다.}

\thm{공사건이 아닌 두 사건 $A, B$에 대하여 다음 조건은 서로 동치이다.
\begin{enumerate}
	\item[(1)] $A, B$가 서로 독립이다. \vspace{-3mm}
	\item[(2)] $\pr(A\cap B) = \pr(A)\pr(B)$
\end{enumerate}}

\thm{공사건이 아닌 두 사건 $A, B$에 대하여 다음이 성립한다.
\begin{center}
	[$A, B$가 독립] $\iff$ [$A^C, B$가 독립] $\iff$ [$A, B^C$가 독립] $\iff$ [$A^C, B^C$가 독립]
\end{center}}

\defn{공사건이 아닌 사건 $A_1, \dots, A_n$에 대하여, $$\pr(A_i\cap A_j) = \pr(A_i)\pr(A_j) \quad \text{for all }1\leq i\neq j\leq n$$
	이 성립하면 사건 $A_1, \dots, A_n$이 \textbf{쌍마다 독립}(pairwise independent)이라고 한다. }

\defn{공사건이 아닌 사건 $A_1, \dots, A_n$에 대하여, $$\pr\left(\bigcap_{i=1}^n A_i\right) = \prod_{i=1}^n \pr(A_i)$$
	이 성립하면 사건 $A_1, \dots, A_n$이 \textbf{상호 독립}(mutually independent)이라고 한다. }

\defn{동일한 시행을 반복할 때, 각 시행에서 일어나는 사건이 서로 독립이면 이러한 시행을 \textbf{독립시행}이라고 한다.}

\thm{\textbf{(독립시행의 확률)} $1$회의 시행에서 사건 $A$가 일어날 확률을 $p$ 라 할 때, 이 시행을 $n$회 반복하는 독립시행에서 사건 $A$가 $r$번 일어날 확률은 다음과 같다. $${n\choose r} p^r(1-p)^r \quad \big(r=0, 1, \dots, n\big)$$}

\thm{(\textbf{전확률공식} - Law of Total Probability) 표본공간 $S$의 분할인 사건 $A_1, \dots, A_n$에 대하여 다음이 성립한다.
$$\pr(B) = \sum_{i=1}^n \pr(B\, |\, A_i)\,\pr(A_i)$$
\textbf{증명.} $\ds \pr(B) = \pr\left( \bigcup_{i=1}^n\left(B\cap A_i\right)\right) = \sum_{i=1}^n \pr(B\cap A_i) = \sum_{i=1}^n\pr(B\, |\, A_i)\,\pr(A_i) $ \qed
}

\thm{(\textbf{베이즈 정리} - Bayes' Theorem) 사건 $A_1, \dots, A_n$이 표본공간 $S$의 분할이고 $\pr(B)>0$ 이면 다음이 성립한다.
$$\pr(A_k \, |\, B) = \frac{\pr(B\,|\,A_k)\,\pr(A_k)}{\ds\sum_{i=1}^n \pr(B\, |\, A_i)\,\pr(A_i)}$$
\textbf{증명.} 조건부확률의 정의와 전확률공식으로부터 자명. \qed
}


\pagebreak

\section{이산확률변수}
\defn{표본공간 위에 정의된 실수값 함수를 \textbf{확률변수}(random variable)라 하고, 확률변수 $X$의 값에 따라 확률이 어떻게 흩어져 있는지를 합이 1인 양수로써 나타낸 것을 $X$의 \textbf{확률분포}(probability distribution)라고 한다.}

\defn{확률변수 $X$가 어떤 값 $x$를 취할 확률은 기호로 $\pr(X=x)$, $a$ 이상 $b$ 이하의 값을 취할 확률은 기호로 $\pr(a\leq X\leq b)$ 와 같이 나타낸다.}

\defn{유한 개이거나, 자연수와 같이 셀 수 있는 값을 취하는 확률변수를 \textbf{이산확률변수}(discrete random variable)라 한다.}

\prob{서로 다른 3개의 동전을 던지는 시행에서 뒷면이 나오는 동전의 개수를 $X$라 할 때, 확률변수 $X$의 확률분표를 표로 나타내어라.\\\vspace{-5mm}
\begin{center}
	\begin{tabular}{c|c|c}
		$X$& ~ & 합계\\
		\hline
		$\pr(X=x)$ & ~\qquad\qquad\qquad\qquad\qquad\qquad\qquad\qquad~ & 1
	\end{tabular}
\end{center}
}

\prob{검은 공 2개와 흰 공 4개가 들어 있는 주머니에서 동시에 3개의 공을 꺼낼 때 나오는 검은 공의 개수를 $X$라 하자. 확률변수 $X$의 확률분포를 표로 나타내어라.}
\\
\defn{이산확률변수 $X$가 취할 수 있는 값이 $x_1, \dots, x_n$ 일 때, $X$의 각 값에 [$X$가 이 값을 취할 확률 $p_1, \dots, p_n$] 을 대응시키는 함수
$$\pr(X=x_i) = p_i\qquad (i=1,\dots, n)$$
를 이산확률변수 $X$의 \textbf{확률질량함수}(probability mass function)라 하며, 확률질량함수는 다음 조건을 만족해야 한다.
\begin{enumerate}
	\item[(1)] $0\leq \pr(X=x_i)\leq 1$
	\item[(2)] $\ds \sum_{i=1}^n \pr(X=x_i) = 1$
	\item[(3)] $\pr(a\leq X\leq b) = \ds \sum_{x=a}^b \pr(X=x)$
\end{enumerate}
}

\prob{5개의 제비 중에 3개의 당첨 제비가 있다. 임의로 뽑은 2개의 제비 중에 있는 당첨 제비의 개수를 $X$라고 할 때, 확률변수 $X$의 확률질량함수를 구하여라.}

\prob{주어진 이산확률변수 $X$에 대해 다음 값을 구하여라.
\begin{center}
	\begin{tabular}{c|c|c|c|c|c}
		$X$ & 0 & 1 & 2 & 3 & 4\\
		\hline
		$\pr(X=x)$ & ~$a$~ & ~$\frac{1}{8}$ ~& ~$b$~ & ~$\frac{1}{4}$~ & ~$\frac{1}{8}$~
	\end{tabular}
\end{center}
\begin{enumerate}
	\item[(1)] $a+b$
	\item[(2)] $\pr(X=1 \cup X=3)$
	\item[(3)] $\pr(0\leq X\leq 2)$
\end{enumerate}
}

\defn{이산확률변수 $X$가 취하는 값이 $x_1, \dots, x_n$일 때,
\begin{itemize}
	\item \textbf{평균}(mean), \textbf{기댓값}(expectation): $\mu = \ex(X) = \ds \sum_{i=1}^n x_i\cdot \pr(X=x_i)$\vspace{-3mm}
	\item \textbf{분산}(variance): $\var(X) = \ex((X-\mu)^2) = \ds \sum_{i=1}^n (x_i-\mu)^2\cdot \pr(X=x_i) $
	\item \textbf{표준편차}(standard deviation): $\sigma(X) = \sqrt{\var(X)}$
\end{itemize}
}

\prob{빨간 공 4개, 흰 공 6개가 들어 있는 주머니에서 3개의 공을 꺼낼 때, 빨간 공이 나오는 개수를 $X$라 한다. $\ex(X), \var(X), \sigma(X)$ 를 구하여라. }

\defn{이산확률변수 $X$와 함수 $g(x)$에 대하여, 다음이 성립한다. $$\ex(g(X)) = \ds \sum_{x\in X} g(x) \cdot \pr(X = x)$$}

\prob{확률변수 $X, Y$에 대하여 다음을 보여라. (단, $a, b$는 상수)
\begin{enumerate}
	\item[(1)] $\ex(aX + b) = a\ex(X) + b$
	\item[(2)] $\var(aX+b) = a^2 \var(X)$
	\item[(3)] $\sigma(aX+b) = |a|\cdot \sigma(X)$
	\item[\textbf{(4)}] $\var(X) = \ex(X^2) - \{\ex(X)\}^2$
\end{enumerate}
}

\prob{연습문제 6.10 의 확률변수 $X$에 대하여 $\ex(5X+4), \var(5X-1), \sigma(5X + 3)$ 을 각각 구하여라.}

\defn{두 확률변수 $X, Y$에 대하여 이들이 취할 수 있는 값들의 모든 순서쌍에 확률이 흩어진 정도를 합이 1인 양수로 나타낸 것을 $X, Y$의 \textbf{결합분포}(joint probability distribution)라 한다.}

\defn{이산확률변수 $X,\:Y$에 대하여 $X$가 취할 수 있는 값을 $x_1, \dots, x_n$, $Y$가 취할 수 있는 값을 $y_1, \dots, y_m$ 이라 하자. 순서쌍 $(x_i, y_j)$ 에 대하여 [결합분포가 이 값을 취할 확률 $p_{ij}$] 을 대응시키는 함수
	$$\pr(X=x_i, Y=y_j) = \pr(X=x_i \cap Y=y_j) = p_{ij}$$
	를 이산확률변수 $X, Y$의 \textbf{결합확률밀도함수}라 하며, 이 함수는 다음 조건을 만족해야 한다.
	\begin{enumerate}
		\item[(1)] $0\leq \pr(X=x_i, Y=y_j)\leq 1$
		\item[(2)] $\ds \sum_{i=1}^n \sum_{j=1}^m \pr(X=x_i, Y=y_j) = 1$\vspace{-3mm}
		\item[(3)] $\pr(a\leq X\leq b, c\leq Y\leq d) = \ds \sum_{x=a}^b \sum_{y=c}^d \pr(X=x, Y=y)$
	\end{enumerate}
}

\defn{이산확률변수 $X, Y$의 결합확률밀도함수 $\pr(X=x_i, Y=y_j)$에서 확률변수 $X$와 $Y$의 확률분포를 얻을 수 있다.
$$\pr(X=x_i) = \sum_{y\in Y} \pr(X=x_i, Y = y), \qquad \pr(Y=y_j) = \sum_{x\in X} \pr(X=x, Y = y_j)$$
이를 각각 $X, Y$의 \textbf{주변확률밀도함수}라 한다.
}

\prob{하나의 주사위를 던져서 나오는 눈의 종류에 따라 상금이 걸린 게임이 있다. A, B 두 사람이 다음과 같은 게임을 한다.\vspace{2mm}\\
$~\quad$ \textbf{A}: 1 또는 2 가 나오면 100원, 3 또는 4가 나오면 200원, 5 또는 6이 나오면 300원\\
$~\quad$ \textbf{B}: 짝수가 나오면 100원, 홀수가 나오면 (눈의 수$\times$100)원\vspace{2mm}\\
이 때, A의 수입을 $X$, B의 수입을 $Y$라 하자. $X, Y$의 결합확률분포, 주변분포를 구하고, 확률변수 $Z=X+Y$로 정의할 때, $Z$의 확률분포와 $\ex(Z)$를 구하여라.\\
\begin{center}
	\begin{tabular}{c|c|c|c|c}
		$x$ \textbackslash  $\text{ }y$ & ~\qquad\qquad~ & ~\qquad\qquad~ & ~\qquad\qquad~ & 행의 합\\
		\hline
		~\qquad~ & ~\qquad\qquad~ & ~\qquad\qquad~ &~\qquad\qquad~ \\
		\hline
		~\qquad~ & ~\qquad\qquad~ & ~\qquad\qquad~ &~\qquad\qquad~ \\
		\hline
		~\qquad~ & ~\qquad\qquad~ & ~\qquad\qquad~ &~\qquad\qquad~ \\
		\hline
		열의 합 & ~\qquad\qquad~ & ~\qquad\qquad~ & ~\qquad\qquad~ & 1
	\end{tabular}
\end{center}
~\\
\begin{center}
	\begin{tabular}{c|c|c|c|c|c|c}
		$z$ &  ~\qquad\quad~ & ~\qquad\quad~ & ~\qquad\quad~ & ~\qquad\quad~ & ~\qquad\quad~ & 합계\\\hline
		$\pr(Z=z)$ & &  & & & & 1
	\end{tabular}
\end{center}
~
}\\

\defn{이산확률변수 $X, Y$와 함수 $g(x, y)$에 대하여, 다음이 성립한다. $$\ex(g(X, Y)) = \ds \sum_{x\in X}\sum_{y\in Y} g(x, y) \cdot \pr(X = x, Y=y)$$}\\

\prob{확률변수 $X, Y$에 대하여 $\ex(aX + bY) = a\ex(X) + b\ex(Y)$ 임을 보여라.}\\

\defn{이산확률변수 $X, Y$가 주어져 있다. 임의의 $x_i\in X, y_j\in Y$에 대해 다음이 성립 하면 확률변수 $X, Y$는 서로 \textbf{독립}이라 한다. $$\pr(X=x_i, Y=y_j) = \pr(X=x_i)\cdot \pr(Y= y_j)$$}

\prob{확률변수 $X, Y$가 서로 독립일 때, 다음이 성립함을 보여라.
\begin{enumerate}
	\item[(1)] $\ex(XY) = \ex(X)\,\ex(Y)$
	\item[(2)] $\var(X \pm Y) = \var(X) + \var(Y)$
\end{enumerate}
}

\prob{위 연습문제의 역은 성립하지 않음을 보여라. (Hint: $-1, 1$을 각각 $1/2$의 확률로 취하는 확률변수 $X$에 대해 $X$와 $X^2$을 고려한다.)}
\section{이산확률분포}
\defn{시행의 결과가 오직 성공(success, $s$) 또는 실패(failure, $f$)뿐이며, 각 시행이 독립이고, 성공의 확률이 $p$ 로 항상 일정한 시행을 \textbf{베르누이 시행}(Bernoulli trial)이라 한다. 성공하면 1, 실패하면 0을 값으로 갖는 확률변수를 \textbf{베르누이 확률변수}(Bernoulli random variable)라 한다.}

\defn{베르누이 확률변수의 확률분포를 \textbf{베르누이 분포}(Bernoulli distribution)라 하고, $X$가 성공 확률이 $p$인 베르누이 분포를 따를 때, $X\sim\mathrm{Berr}(p)$ 와 같이 나타낸다.\footnote{$X$가 분포 $\mc{A}$ 를 따를 때, $X\sim \mc{A}$ 와 같이 표기한다.}}

\prob{$X\sim \mathrm{Berr}(p)$ 일 때, $X$의 확률분포표를 구하고, $\ex(X)$와 $\var(X)$를 구하여라.}\\\\\\

\defn{한 번의 시행에서 사건 $A$가 일어날 확률이 $p$로 일정할 때, $n$번의 독립시행에서 사건 $A$가 일어나는 횟수를 $X$라고 하면 확률변수 $X$의 확률분포를 \textbf{이항분포}(binomial distribution)라 하고 기호로 $\mr{B} (n, p)$ 와 같이 나타낸다.}

\defn{성공 확률이 $p$인 베르누이 시행을 $n$번 독립적으로 반복 시행할 때, 성공 횟수의 분포를 \textbf{이항분포}라 한다. 즉, $i=1, \dots, n$ 에 대하여 $X_i\sim_{i.i.d} \mathrm{Berr}(p)$일 때,\footnote{$i.i.d$: independent and identically distributed.} 이항분포는 $n$개의 베르누이 확률변수의 합으로 정의된다.\footnote{따라서 $\mr{Berr}(p) = \mr{B}(1, p)$ 이다.}
	$$\sum_{i=1}^n X_i= X \sim \mr{B}(n, p)$$ }\\

\defn{$X\sim \mr{B}(n, p)$일 때, $X$의 확률질량함수는 다음과 같다.
$$\pr(X=x) = {n\choose x}p^x(1-p)^{n-x} \qquad (x=0, \dots, n)$$ }

\prob{다음 확률변수 $X$가 이항분포를 따르는지 조사하시오.
\begin{enumerate}
	\item[(1)] 10개의 동전을 동시에 던질 때 뒷면이 나오는 동전의 개수 $X$
	\item[(2)] 검정 구슬 4개와 흰 구슬 2개 중에서 차례로 2개의 구슬을 꺼낼 때 나오는 흰 구슬의 개수 $X$
	\item[(3)] 4지선다형 문제 12개에 임의로 답할 때 정답의 개수 $X$ 
\end{enumerate}
}

\prob{타율이 0.2인 야구 선수가 10번의 타석에서 안타를 친 횟수를 $X$라 하자. $\pr(X\leq 9)$ 의 값을 구하여라.}\\

\prob{$X\sim \mr{B}(n, p)$ 이면, $\ex(X) = np$, $\var(X) = np(1-p)$ 임을 보여라.}\\\\


\prob{두 사람 A, B가 게임을 한다. 매 회 동전을 던져 앞면이 나오면 A가 이기고, 뒷면이 나오면 B가 이긴다. 10회 게임을 할 때, A가 이긴 횟수를 $X$, B가 이긴 횟수를 $Y$라 하자. $\ex(X), \: \ex(Y)$ 를 구하여라.}\\

\prob{$X\sim \mr{B}(100, p)$ 라 하자. $X$의 분산이 최대일 때, $\ex(X)$의 값을 구하여라.}\\

\defn{특성값 1의 개수가 $D$, 0의 개수가 $N-D$ 인 크기 $N$의 유한 모집단에서 크기 $n$인 랜덤 표본을 뽑을 때, 표본에서 1의 개수를 $X$라 하자. 이 때, 확률변수 $X$가 따르는 분포를 \textbf{초기하분포}(hypergeometric distribution)라 하고, 기호로는 $X\sim \mr{H}(N, D, n)$으로 나타낸다. 초기하분포의 확률질량함수는 다음과 같이 주어진다. $$\pr(X = x) = \frac{\ds{D\choose x}{N-D \choose n-x}}{\ds{N \choose n}} \qquad \big(\text{단, }\max\{0, n-N+D\}\leq x\leq \min\{n, D\}\big)$$}

\thm{$X\sim \mr{H}(N, D, n)$ 일 때, 다음이 성립한다. $$\ex(X) = np,\quad \var(X) = np\left(1-p\right)\frac{N-n}{N-1} \quad \left(p=\frac{D}{N}\right)$$}

\thm{$X\sim \mr{H}(N, D, n)$ 일 때, $N \gg n$ 이면 $X$는 근사적으로 $\mr{B}(n, D/N)$ 를 따른다.}

\defn{성공 확률이 $p$인 베르누이 시행을 반복하여 최초로 성공할 때 까지의 시행 횟수를 $X$라 하자. 이 때, 확률변수 $X$는 \textbf{기하분포}(geometric distribution)을 따른다. 기하분포의 확률질량함수는 다음과 같이 주어진다. $$\pr(X=k) = p(1-p)^{k-1} \qquad (k=0, 1, \dots)$$}

\prob{성공 확률이 $p$인 기하분포를 따르는 확률변수 $X$에 대하여 다음이 성립함을 보여라. $$\ex(X) = \frac{1}{p}, \quad \var(X) = \frac{1-p}{p^2}$$}\\

\thm{실수열 $\{p_n\}$ ($0\leq p_i\leq 1$ for all $i$) 에 대하여 $\ds\lim_{n\rightarrow \infty} np_n = \lambda$ 라 하면 다음이 성립한다. $$\lim_{n\rightarrow \infty} {n\choose k}p_n^k(1-p_n)^{n-k} = e^{-\lambda}\frac{\lambda^k}{k!}$$} 

\defn{정해진 시간 안에 어떤 사건이 일어날 횟수에 대한 기댓값을 $\lambda$라 할 때, 그 사건이 일어난 횟수를 $X$라 하자. 이 때, 확률변수 $X$는 \textbf{포아송 분포}(Poisson distribution)를 따르며, 기호로는 $X\sim \mr{Poi}(\lambda)$ 로 나타낸다. 포아송 분포의 확률질량함수는 다음과 같이 주어진다.
	$$\pr(X=k) = e^{-\lambda}\frac{\lambda^k}{k!}\qquad(k=0, 1, \dots)$$
}

\thm{$X\sim \mr{B}(n, p)$ 일 때, $n$이 충분히 크면 $X\sim \mr{Poi}(np)$ 이다.\footnote{대략 $n\geq 100, np\leq 10$ 이면 근사할 수 있다.}}

\prob{$X\sim \mr{Poi}(\lambda)$ 일 때, $\ex(X) = \lambda$, $\var(X)=\lambda$ 임을 보여라.}\\


\prob{오후 3시부터 4시에 어느 병원에 도착하는 손님이 평균적으로 6.5명이라 하자. 오늘 오후 3시부터 4시 사이에 도착하는 손님 수에 대한 확률질량함수를 구하고, 도착한 손님이 4명일 확률, 최대 2명일 확률을 각각 구하여라.}\\\\\\\\\\


\pagebreak

\section{연속확률변수}
\defn{어떤 구간에 속하는 모든 실수 값을 취할 수 있는 확률변수를 \textbf{연속확률변수}(continuous random variable)라 한다. 연속확률변수 $X$가 구간 $[\alpha, \beta]$에 속하는 모든 실수 값을 취하고, $$\pr(a\leq X\leq b) = \int_a^b f(x)dx \quad (\alpha \leq a\leq x\leq b\leq \beta)$$ 와 같이 나타낼 수 있을 때, 함수 $f(x)$를 $X$의 \textbf{확률밀도함수}(probability density function)라 하며, 확률밀도함수는 다음 조건을 만족해야 한다.
\begin{enumerate}
	\item[(1)] $f(x) \geq 0$
	\item[(2)] $\ds \int_\alpha^\beta f(x)dx= 1$
	\item[(3)] $\alpha \leq a\leq x\leq b\leq \beta$ 일 때, $$
	\pr(a\leq X\leq b) = \pr(X\leq b) - \pr(X < a) = \int_\alpha^b f(x)dx \, - \,\int_\alpha^a f(x)dx = \int_a^b f(x)dx$$
\end{enumerate}
}

\defn{연속확률변수 $X$가 구간 $[\alpha, \beta]$에 속하는 모든 실수 값을 취할 때,
\begin{itemize}
	\item \textbf{평균}(mean), \textbf{기댓값}(expectation): $\mu = \ex(X) = \ds \int_\alpha^\beta xf(x)dx$\vspace{-3mm}
	\item \textbf{분산}(variance): $\var(X) = \ex((X-\mu)^2) = \ds \int_\alpha^\beta (x-\mu)^2f(x)dx $
	\item \textbf{표준편차}(standard deviation): $\sigma(X) = \sqrt{\var(X)}$
\end{itemize}
}

\prob{연속확률변수 $X$의 확률밀도함수가 $f(x) = ax \: (0\leq x\leq 2)$ 일 때, 상수 $a$의 값과 $\pr(0.5\leq X\leq 1)$의 값을 구하고, $\ex(X)$와 $\var(X)$의 값을 구하여라.}

\defn{연속확률변수 $X$가 모든 실수 값을 취하고, 확률밀도함수 $f(x)$가 다음과 같이 주어질 때, $$f(x) = \frac{1}{\sqrt{2\pi} \sigma} e^{-\frac{(x-\mu)^2}{2\sigma^2}} \qquad (-\infty < x<\infty)$$ $X$의 확률분포를 \textbf{정규분포}(normal distribution)라 하고, 평균이 $\mu$, 분산이 $\sigma^2$인 정규분포를 기호로 $\mc{N}(\mu, \sigma^2)$와 같이 나타낸다.}

\thm{서로 독립인 확률변수 $X_i \sim \mc{N}(\mu_i, \sigma_i^2)$, 상수 $c_i$ $(i = 1,\dots, k)$ 에 대하여 $$X = \sum_{i=1}^k c_iX_i \sim \mc{N}\left(\sum_{i=1}^k c_i\mu_i,\: \sum_{i=1}^n c_i^2 \sigma_i^2\right)$$ 이 성립한다.}

\begin{center}
	\begin{tikzpicture}
	\begin{axis}[
		title = {표준정규분포 $\mc{N}(0, 1)$},
		every axis plot post/.append style={
		mark=none,domain=-4:4,samples=50,smooth},
	axis x line*=bottom,
	axis y line=none, enlargelimits=upper]
	\addplot {gauss(0,1)};
	\addlegendentry{$\mc{N}(0, 1)$}
	\end{axis}
	\end{tikzpicture}
\end{center}


\defn{$\mc{N}(0, 1)$ 을 \textbf{표준정규분포}(standard normal distribution)라 하고, 확률밀도함수 $f(z)$는 다음과 같다. $$f(z) = \frac{1}{\sqrt{2\pi}}e^{-\frac{z^2}{2}} \qquad (-\infty < z <\infty)$$}

\thm{(\textbf{정규분포의 표준화}) $X\sim \mc{N}(\mu, \sigma^2)$ 일 때, $$Z = \frac{X-\mu}{\sigma} \sim \mc{N}(0, 1)$$가 성립하고, 이를 \textbf{정규분포의 표준화}(standardization)라 한다. 이렇게 표준화된 값을 \textbf{$z$-점수}($z$-score)라 하고, 다음이 성립한다. $$\pr(a\leq X\leq b) = \pr\left(\frac{a-\mu}{\sigma} \leq Z\leq \frac{b-\mu}{\sigma}\right)$$}\\

\thm{표준정규분포 $\mc{N}(0, 1)$ 의 확률밀도함수 $f(z)$는 다음과 같은 성질을 갖는다.
\begin{itemize}
	\item 곡선과 $x$축 사이의 넓이는 $1$이다.
	\item 직선 $x=0$ 에 대하여 대칭이다.
\end{itemize}이로부터 다음이 성립함을 알 수 있다. 임의의 실수 $a, b$ 에 대하여 
\begin{enumerate}
	\item[(1)] $\pr(Z\geq a) = \pr(Z \leq -a)$.
	\item[(2)] $\pr(Z\geq a) = 1-\pr(Z<a)$
	\item[(3)] $\pr(a\leq Z\leq b) = \pr(Z\leq b) - \pr(Z < a)$
\end{enumerate}
}

\prob{표준정규분포표가 주어져 있다. $X\sim \mc{N}(27, 4^2)$ 일 때, 다음을 구하여라.
\begin{center}
	\begin{tabular}{c|c}
		$z$ & $\pr(0\leq Z\leq z)$ \\ \hline
		0.5 & 0.1915 \\
		1.0 & 0.3413 \\
		1.5 & 0.4332 \\
		2.0 & 0.4772
	\end{tabular}
\end{center}
\begin{enumerate}
	\item[(1)] $\pr(X\leq 21)$
	\item[(2)] $\pr(29\leq X\leq 35)$
	\item[(3)] $\pr(X\geq 25)$
\end{enumerate}
}


\defn{$\mc{N}(0, 1)$ 의 $100(1-\alpha)$ 백분위수를 $z_\alpha$ 로 나타낸다. 즉 $\pr(Z\geq z_\alpha) = \alpha$ 이다.\footnote{\textbf{상방백분위수}라고도 한다.}}

\prob{$\pr(Z<1.96) = 0.975$, $\pr(Z < 2.58) = 0.995$ 일 때, $z_{0.025}, z_{0.005}$ 의 값을 구하여라.}\\

\thm{(68.26-95.44-99.74 Rule) $X\sim \mc{N}(\mu, \sigma^2)$ 일 때,
\begin{itemize}
	\item 68.26\% 의 관측값들이 $\left[\mu - \sigma , \mu +\sigma\right]$ 에 있다. $\pr\left(\left|X-\mu\right|<\sigma\right) = 0.6826$.
	\item 95.44\% 의 관측값들이 $\left[\mu - 2\sigma , \mu +2\sigma\right]$ 에 있다. $\pr\left(\left|X-\mu\right|<2\sigma\right) = 0.9544$.
	\item 99.74\% 의 관측값들이 $\left[\mu - 3\sigma , \mu + 3\sigma\right]$ 에 있다. $\pr\left(\left|X-\mu\right|<3\sigma\right) = 0.9974$.
\end{itemize}
}

\prob{전세계 사람들의 IQ는 평균이 100, 표준편차가 16인 정규분포를 따른다고 한다. IQ가 116, 132, 148인 사람은 각각 IQ 상위 몇 \% 인지 구하여라.}\\\\

\prob{대수능 모의평가에서 어느 고등학교 3학년 학생 500명의 수학 성적이 평균 70점, 표준편차 20점인 정규분포를 따른다고 한다. 300등을 한 학생의 점수를 구하여라.
\begin{center}
	\begin{tabular}{c|c}
		$z$ & $\pr(0\leq Z\leq z)$ \\ \hline
		0.25 & 0.1 \\
		0.52 & 0.2 \\
		1.28 & 0.4
	\end{tabular}
\end{center}
}\\

\prob{다음은 9등급제의 산출 방식 중 일부이다.
\begin{center}
	\begin{tabular}{c|c|c}
		등급 & $z$ & $\pr(Z\geq z)$ \\ \hline
		1 & 1.75 & 0.04 \\ 2 & 1.25 & 0.11 \\ 3 & 0.75 & 0.23 \\ 4 & 0.25 & 0.40
	\end{tabular}
\end{center}
어떤 시험의 평균이 50점이고 표준편차가 18점일 때, 각 등급컷 점수를 구하여라.
}\\\\\\\\

\thm{(\textbf{드 무아브르-라플라스의 정리}) $X\sim \mr{B}(n, p)$ 일 때, $n$이 충분히 크면\footnote{일반적으로, $np\geq 5, n(1-p)\geq 5$ 이면 근사한다.} $X$는 근사적으로 $\mc{N}(np, np(1-p))$ 를 따른다.}

\thm{(\textbf{연속성 수정} - continuity correction) 연속확률분포를 이용하여 이산확률분포의 확률을 근사시킬 때, 근사의 정밀도를 높이는데 사용한다. $X\sim \mr{B}(n, p)$ 일 때, $$\pr(a\leq X\leq b) \approx \pr\left(\frac{a-np \mathbf{-0.5}}{\sqrt{np(1-p)}} \leq Z \leq\frac{a-np \mathbf{+0.5}}{\sqrt{np(1-p)}} \right)$$}

\prob{현재 20살인 사람이 45년 후 살아있을 확률이 0.8 이라고 한다. 20살인 사람 500명을 임의로 추출했을 때, 다음 값을 식으로 표현하여라.
\begin{enumerate}
	\item[(1)] 45년 후, 정확히 390명이 살아있을 확률
	\item[(2)] 45년 후, 375명 이상 425명 이하의 사람들이 살아있을 확률
\end{enumerate}
}
\\\\\\\\\\\\
\pagebreak


\end{document}
