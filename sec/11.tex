\section{분포에 대한 추론}
통계적 추론을 필요로 하는 많은 문제들은 하나의 모집단에 관한 것이라기 보다는 여러 모집단을 비교하기 위한 경우가 더 많다. \\\\
\textbf{예제}. 두 종류의 진통제에 대한 상대적 효과의 척도로서, 복용 후 숙면할 수 있는 정도를 비교하려고 한다. 비교 실험에 참여하는 환자 6명을 랜덤추출 했으나 각 환자들의 건강상태에는 상당한 차이가 있으므로, 각 환자에게 두 종류의 진통제를 각각 1회씩 복용하게 하여 숙면시간의 차이를 이용하여 두 진통제의 효과를 비교하기로 하였다.\\
\begin{center}
	\begin{tabular}{c|c|c|c|c|c|c}
		환자 & 1& 2&3&4&5&6 \\\hline
		진통제 A&4.8&4.0&5.8&4.9&5.3&7.4 \\\hline
		진통제 B&4.0&4.2&5.2&4.9&5.6&7.1
	\end{tabular}
\end{center}~\\
두 진통제의 효과에 차이가 있는가?\\\\

\defn{~
\begin{itemize}
	\item \textbf{실험 단위}(experimental unit): 비교의 목적을 위해 그 매개체로 사용되는 대상
	\item \textbf{처리}(treatment): 실험 단위에 적용되어 특성치를 결정지어주는 것
	\item \textbf{처리 효과}(treatment effect): 비교 대상인 특성치
	\item \textbf{대응비교} 또는 \textbf{쌍체비교}(paired comparison): 두 모집단의 평균을 비교할 때 실험 단위를 동질적인 쌍으로 묶은 다음, 각 쌍에 두 처리를 임의로 적용하고, 각 쌍에서 모은 관측값의 차로 처리효과의 차에 관한 추론을 하는 방법
\end{itemize}
}

\prob{위 \textbf{예제}의 상황에 사용되는 방법이 대응비교이다. 예제에서 실험단위, 처리, 처리효과를 찾아보아라.}\\

\thm{\textbf{대응비교의 자료구조 및 모형}
\begin{itemize}
	\item 자료구조: 랜덤표본 $(X_1, Y_1), \dots, (X_n, Y_n)$
	\begin{center}
		\begin{tabular}{c|c|c|c}
			쌍&처리 1&처리 2&처리효과의 차\\\hline
			1 & $X_1$ & $Y_1$ & $D_1 = X_1-Y_1$\\\hline 
			2 & $X_2$ & $Y_2$ & $D_2 = X_2-Y_2$\\\hline 
			$\vdots$ & $\vdots$ & $\vdots$ & $\vdots$\\\hline
			$n$ & $X_n$ & $Y_n$ & $D_n = X_n-Y_n$
		\end{tabular}
	\end{center}
	\item 처리효과: $\ex(X_i) = \mu_1$, $\ex(Y_i) = \mu_2$
	\item 처리효과의 차: $\delta = \mu_1-\mu_2$
	\item 차에 대한 가정: $D_i = X_i-Y_i \sim_{i.i.d}\mc{N}(\delta, \sigma_D^2)$, $i = 1, \dots, n$. ($\sigma_D$ 는 미지의 값)
	\item $\mu_1-\mu_2$ 에 대한 추정량: $\widehat{\mu_1-\mu_2} = \hat{\delta} = \overline{D} = \ds \frac{1}{n}\sum_{i=1}^{n}D_i$
	\item $\mu_1-\mu_2$에 대한 $100(1-\alpha)\%$ 신뢰구간: $$\left(\overline{d} - t_{\alpha/2}(n-1)\cdot\frac{s_D}{\sqrt{n}},\: \overline{d}+t_{\alpha/2}(n-1)\cdot\frac{s_D}{\sqrt{n}} \right)$$
	단, $\ds s_D^2 = \frac{1}{n-1}\sum_{i=1}^n(d_i-\overline{d})^2$.
\end{itemize}}\\
\textbf{참고}. $D_i\sim_{i.i.d}\mc{N}(\delta, \sigma_D^2)$ 이므로 $\overline{D} \sim \mc{N}(\delta, \sigma_D^2/n)$ 이다. 모표준편차를 알지 못하므로, 추정을 위해 표본표준편차를 사용하며, 분포는 $t$-분포를 사용한다. 표준화하면, $$T = \frac{\overline{D}-\delta}{S_D/\sqrt{n}} \sim t(n-1)$$ 가 됨을 알 수 있다. 이로부터 신뢰구간을 구하면 된다.\\\\

\prob{위 \textbf{예제}에서 주어진 자료를 이용하여 두 진통제 $A, B$에 의한 평균 숙면시간의 차이에 대한 95\% 신뢰구간을 구하여라.}\\

\thm{\textbf{대응비교에 의한 두 모평균의 비교} - $t$ 검정
	\begin{itemize}
		\item 가정: 대응비교의 자료구조에서의 가정과 동일
		\item 귀무가설 $H_0$: $\mu_1-\mu_2 = \delta_0$
		\item \textbf{검정통계량}: $\ds T = \frac{\overline{D} - \delta_0}{S_D/\sqrt{n}}\sim_{H_0}t(n-1)$,\quad \textbf{관측값}: $\ds t = \frac{\overline{d}-\delta_0}{s_D/\sqrt{n}}$ ($s_D$: 표본표준편차)
		\item 대립가설의 형태에 따른 기각역과 유의확률
		\begin{center}
			\begin{tabular}{c|c|c}
				대립가설 & 유의수준 $\alpha$에서의 기각역 & 유의확률\\ \hline
				$H_1$: $\mu_1-\mu_2 >\delta_0$ & $t\geq t_{\alpha}(n-1)$ & $\pr\left(T\geq t\right)$\\ 
				$H_1$: $\mu_1-\mu_2 <\delta_0$ & $t\leq -t_{\alpha}(n-1)$ & $\pr\left(T\leq t\right)$\\ 
				$H_1$: $\mu_1-\mu_2 \neq \delta_0$ & $\left|t\right|\geq t_{\alpha/2}(n-1)$ & $\pr\left(\left|T\right|\geq \left|t\right|\right)$
			\end{tabular}
		\end{center}
	\end{itemize}
}

\prob{\textbf{예제}에서 주어진 자료를 활용하여 두 진통제의 효과에 차이가 있는지 유의수준 1\%에서 검정하여라. 그리고 유의확률도 구하여라.}
\\
두 가지 처리를 비교하는 경우에는 대응비교와는 달리 실험 단위를 두 그룹으로 나누어 그룹 별로 서로 다른 처리를 적용하여 그 결과를 이용하여 두 처리를 비교할 수도 있다. 한 실험 단위에 두 처리를 모두 적용하기 어려운 경우에는 대응비교를 사용할 수 없다.\\

\thm{\textbf{이표본에 의한 모평균의 비교 - 가정}
\begin{itemize}
	\item 모형 가정: 각 그룹에서의 관측값은 각 모집단에서의 랜덤 표본이고, 서로 다른 그룹에서의 관측값들은 독립적으로 관측된 것이다.
	\item 분포 가정: $X_1, \dots, X_{n_1}$ 와 $Y_1, \dots, Y_{n_2}$ 는 각각 $\mc{N}(\mu_1, \sigma_1^2)$, $\mc{N}(\mu_2, \sigma_2^2)$ 의 랜덤표본
\end{itemize}
}

\thm{$\mu_1-\mu_2$ 에 대한 추정
\begin{itemize}
	\item 추정량: $\widehat{\mu_1-\mu_2} = \widehat{\mu_1} - \widehat{\mu_2} = \overline{X} - \overline{Y}$
	\item 추정값: $\widehat{\mu_1 - \mu_2} = \overline{x} - \overline{y}$
	\item 표본분포: $\overline{X} \sim \mc{N}\left(\mu_1, \dfrac{\sigma_1^2}{n_1}\right)$, $\overline{Y} \sim \mc{N}\left(\mu_2, \dfrac{\sigma_2^2}{n_2}\right)$, $\overline{X}-\overline{Y} \sim \mc{N}\left(\mu_1-\mu_2, \ds \frac{\sigma_1^2}{n_1} + \frac{\sigma_2^2}{n_2}\right)$ 
\end{itemize}
\textbf{참고}. 모형 가정에 의해 표본이 서로 독립이므로, $$\var(\overline{X} - \overline{Y}) = \var(\overline{X}) + \var(\overline{Y}) =  \frac{\sigma_1^2}{n_1} + \frac{\sigma_2^2}{n_2}$$
따라서 표준화하면, $$\frac{\left(\overline{X} - \overline{Y}\right) - \left(\mu_1-\mu_2\right)}{\sqrt{\sigma_1^2/n_1 + \sigma_2^2/n_2}} \sim \mc{N}(0, 1)$$
}

\thm{\textbf{이표본에 의한 모평균의 비교 \underline{($\sigma_1, \sigma_2$ 를 아는 경우)}}\\
	표본의 크기 $n_1, n_2$ 가 충분히 큰 경우, 정규 모집단 가정이 생략 가능 (중심극한정리)
\begin{itemize}
	\item $\mu_1-\mu_2$ 에 대한 $100(1-\alpha)\%$ 신뢰구간: $(\overline{x}-\overline{y}) \pm z_{\alpha/2}\ds\sqrt{\frac{\sigma_1^2}{n_1} + \frac{\sigma_2^2}{n_2}}$
	\item $H_0$: $\mu_1-\mu_2 = \delta_0$ 에 대한 검정
		\begin{itemize}
			\item 검정통계량: $Z = \ds \frac{\overline{X} - \overline{Y} - \delta_0}{\ds \sqrt{ \frac{\sigma_1^2}{n_1} + \frac{\sigma_2^2}{n_2}}} \sim_{H_0} \mc{N}(0, 1)$
			\item 관측값: $z = \ds \frac{\overline{x} - \overline{y} - \delta_0}{\ds \sqrt{\frac{\sigma_1^2}{n_1} + \frac{\sigma_2^2}{n_2}}}$
		\end{itemize}
	\item 대립가설의 형태에 따른 기각역과 유의확률
	\begin{center}
		\begin{tabular}{c|c|c}
			대립가설 & 유의수준 $\alpha$에서의 기각역 & 유의확률\\ \hline
			$H_1$: $\mu_1-\mu_2 >\delta_0$ & $z\geq z_{\alpha}$ & $\pr\left(Z\geq z\right)$\\ 
			$H_1$: $\mu_1-\mu_2 <\delta_0$ & $z\leq -z_{\alpha}$ & $\pr\left(Z\leq z\right)$\\ 
			$H_1$: $\mu_1-\mu_2 \neq \delta_0$ & $\left|z\right|\geq z_{\alpha/2}$ & $\pr\left(\left|Z\right|\geq \left|z\right|\right)$
		\end{tabular}
	\end{center}
\end{itemize}
}\\\\
그러나 실제 문제에서는 모표준편차 $\sigma_1, \sigma_2$ 를 모르는 경우가 많다. 따라서 모표준편차 대신 그 추정량인 표본표준편차를 사용하여 스튜던트화된 표본평균의 차인 $$T = \frac{\left(\overline{X} - \overline{Y}\right) - \left(\mu_1-\mu_2\right)}{\sqrt{S_1^2/n_1 + S_2^2/n_2}}$$ 를 사용하여 추론을 하게 된다. \\\\

\thm{\textbf{이표본에 의한 모평균의 비교 \underline{($\sigma_1$, $\sigma_2$ 를 모르고, $\sigma_1 \neq \sigma_2$, $n_1, n_2 \gg 1$)}}\\
	표본의 크기가 충분히 크면 $T$의 분포가 표준정규분포에 가까워진다.
\begin{itemize}
	\item $\mu_1-\mu_2$ 에 대한 $100(1-\alpha)\%$ 신뢰구간: $(\overline{x}-\overline{y}) \pm z_{\alpha/2}\ds\sqrt{\frac{s_1^2}{n_1} + \frac{s_2^2}{n_2}}$
		\item $H_0$: $\mu_1-\mu_2 = \delta_0$ 에 대한 검정
	\begin{itemize}
		\item 검정통계량: $T = \ds \frac{\overline{X} - \overline{Y} - \delta_0}{\ds \sqrt{ \frac{S_1^2}{n_1} + \frac{S_2^2}{n_2}}} \sim_{H_0} \mc{N}(0, 1)$
		\item 관측값: $t = \ds \frac{\overline{x} - \overline{y} - \delta_0}{\ds \sqrt{\frac{s_1^2}{n_1} + \frac{s_2^2}{n_2}}}$
	\end{itemize}
	\item 대립가설의 형태에 따른 기각역과 유의확률\footnote{표본의 크기가 크므로, 근사적으로 $T\sim \mc{N}(0, 1)$ 임에 주의한다.}
	\begin{center}
		\begin{tabular}{c|c|c}
			대립가설 & 유의수준 $\alpha$에서의 기각역 & 유의확률\\ \hline
			$H_1$: $\mu_1-\mu_2 >\delta_0$ & $t\geq z_{\alpha}$ & $\pr\left(T\geq t\right)$\\ 
			$H_1$: $\mu_1-\mu_2 <\delta_0$ & $t\leq -z_{\alpha}$ & $\pr\left(T\leq t\right)$\\ 
			$H_1$: $\mu_1-\mu_2 \neq \delta_0$ & $\left|t\right|\geq z_{\alpha/2}$ & $\pr\left(\left|T\right|\geq \left|t\right|\right)$
		\end{tabular}
	\end{center}
\end{itemize}
}

\prob{지역 환경에 따라 학력에 차이가 있는가를 알아보기 위하여, 두 도시의 중학교 1학년 학생 중에서 각각 90명과 100명을 랜덤추출하여 동일한 시험을 시행한 결과가 다음과 같았다. 두 도시의 중학교 1학년 학생 전체의 평균 성적에 차이가 있는지 유의수준 1\%에서 검정하고, 유의확률을 구하여라. 그리고 평균 성적의 차이에 대한 신뢰수준 99\%의 신뢰구간도 구하여라.
\begin{center}
	\begin{tabular}{c|c|c}
		& 도시 1 & 도시 2\\ \hline
		표본 크기 & 90 & 100\\ \hline
		평균 & 76.4 & 81.2 \\ \hline
		표준편차 & 8.2 & 7.6 \\ \hline
	\end{tabular}
\end{center}
}\\\\
~\\
\thm{\textbf{이표본에 의한 모평균의 비교 \underline{($\sigma_1$, $\sigma_2$ 를 모르고, $\sigma_1 \neq \sigma_2$)}}\\
	$n_1, n_2$ 가 충분히 크지 않고(5 이상), 정규모집단인 경우이다.
\begin{itemize}
	\item $df = \ds \frac{\left(\dfrac{s_1^2}{n_1} + \dfrac{s_2^2}{n_2}\right)^2}{\dfrac{(s_1^2/n_1)^2}{n_1-1} + \dfrac{(s_2^2/n_2)^2}{n_2-1}}$ \quad (Welch–Satterthwaite Equation)
	\item $\mu_1-\mu_2$ 에 대한 $100(1-\alpha)\%$ 신뢰구간: $(\overline{x}-\overline{y}) \pm t_{\alpha/2}(df) \ds \sqrt{ \frac{s_1^2}{n_1} + \frac{s_2^2}{n_2}}$
	\item $H_0$: $\mu_1-\mu_2 = \delta_0$ 에 대한 검정
	\begin{itemize}
		\item 검정통계량: $T = \ds \frac{\overline{X} - \overline{Y} - \delta_0}{\ds \sqrt{ \frac{S_1^2}{n_1} + \frac{S_2^2}{n_2}}} \sim_{H_0} t(df)$
		\item 관측값: $t = \ds \frac{\overline{x} - \overline{y} - \delta_0}{\ds \sqrt{\frac{s_1^2}{n_1} + \frac{s_2^2}{n_2}}}$
	\end{itemize}
	\item 대립가설의 형태에 따른 기각역과 유의확률
	\begin{center}
		\begin{tabular}{c|c|c}
			대립가설 & 유의수준 $\alpha$에서의 기각역 & 유의확률\\ \hline
			$H_1$: $\mu_1-\mu_2 >\delta_0$ & $t\geq t_{\alpha}(df)$ & $\pr\left(T\geq t\right)$\\ 
			$H_1$: $\mu_1-\mu_2 <\delta_0$ & $t\leq -t_{\alpha}(df)$ & $\pr\left(T\leq t\right)$\\ 
			$H_1$: $\mu_1-\mu_2 \neq \delta_0$ & $\left|t\right|\geq t_{\alpha/2}(df)$ & $\pr\left(\left|T\right|\geq \left|t\right|\right)$
		\end{tabular}
	\end{center}
\end{itemize}
}\\\\

\prob{16 마리의 쥐를 대상으로 진행한 다음 실험에 대하여 질산칼륨의 과다 섭취가 성장을 저해하는지 유의수준 5\% 에서 검정하여라.
\begin{center}
	\begin{tabular}{c|c|c}
		& 질산칼륨 섭취 & 규정식 섭취 \\ \hline
		표본 크기 & 9 & 7 \\ \hline
		평균 & 15.07 & 19.27 \\ \hline
		표준편차 & 3.56 & 8.05 \\ \hline
		
	\end{tabular}
\end{center}
}~\\\\
\\
위에서 소개한 방법들은 전부 근사적인 방법이다. 또한, 두 모집단의 분산이 다르다는 가정 하에서 사용할 수 있는 방법들이었다. 이와 같이 \textbf{모집단의 분산이 다를 때} 사용하는 $t$ 검정을 \textbf{비합동 이표본 $t$ 검정}(nonpooled two sample $t$-test)이라 한다. 두 \textbf{모집단의 분산이 같은 경우} 사용하는 더욱 효율적인 추론 방법인 \textbf{합동 이표본 $t$-검정}(pooled two sample $t$-test)에 대해 알아보자.\\

\thm{\textbf{공통분산의 추정 \underline{($\sigma_1^2 = \sigma_2^2=\sigma^2$)}}\\
	두 표본분산 $S_1^2$, $S_2^2$ 의 자유도인 $n_1-1$, $n_2-1$ 을 가중치로 하여 이들의 가중치 평균인 $$S_p^2 = \ds \frac{(n_1-1)S_1^2 + (n_2-1)S_2^2}{n_1+n_2-2}$$
	를 공통분산 $\sigma^2$ 의 추정량으로 생각할 수 있다. 이를 \textbf{합동 표본 분산}(pooled sample variance)라고 하고 $S_p^2$ 으로 나타내며, 자유도는 $n_1+n_2-2$ 이다.
\begin{center}
	\begin{tabular}{c|c|c|c}
		& 모집단 1 & 모집단 2 & 합동표본분산\\ \hline 
		&&&\\[-1em]
		표본분산 & $S_1^2$ & $S_2^2$ & $S_p^2 = \ds \frac{(n_1-1)S_1^2 + (n_2-1)S_2^2}{n_1+n_2-2}$\\
		&&&\\[-1em] \hline
		자유도 & $n_1-1$ & $n_2-1$ & $n_1+n_2-2$ \\\hline
	\end{tabular}
\end{center}
}\\\\
두 모집단의 분산이 같을 때, $\var(\overline{X} -\overline{Y})  = \ds \frac{\sigma^2}{n_1} + \frac{\sigma^2}{n_2}$ 이다. 
표본평균의 차를 표준화하면
$$\frac{\left(\overline{X} - \overline{Y}\right) - \left(\mu_1-\mu_2\right)}{\sigma \ds \sqrt{\frac{1}{n_1} + \frac{1}{n_2}}}$$
를 얻고, 모분산 $\sigma^2$ 대신 그 추정량인 $S_p^2$ 를 이용하여 스튜던트화 하면
$$T = \frac{\left(\overline{X} - \overline{Y}\right) - \left(\mu_1-\mu_2\right)}{S_p \ds \sqrt{\frac{1}{n_1} + \frac{1}{n_2}}} \sim t(n_1+n_2-2)$$ 를 얻는다. 이를 이용하여 모평균을 비교할 수 있다.\\

\thm{\textbf{이표본에 의한 모평균의 비교 \underline{($\sigma_1$, $\sigma_2$ 를 모르고, $\sigma_1 = \sigma_2$)}}\\
공통분산의 정규모집단인 경우이다.
\begin{itemize}
	\item $\mu_1-\mu_2$ 의 $100(1-\alpha)\%$ 신뢰구간: $(\overline{x} -\overline{y}) \pm t_{\alpha/2}(n_1+n_2-2) s_p \ds \sqrt{\frac{1}{n_1} + \frac{1}{n_2}}$
	\item $H_0$: $\mu_1-\mu_2 = \delta_0$ 에 대한 검정
	\begin{itemize}
		\item 검정통계량: $T = \ds \frac{\overline{X} - \overline{Y} - \delta_0}{\ds S_p \sqrt{ \frac{1}{n_1} + \frac{1}{n_2}}} \sim_{H_0} t(n_1+n_2 - 2)$
		\item 관측값: $t = \ds \frac{\overline{x} - \overline{y} - \delta_0}{\ds s_p \sqrt{\frac{1}{n_1} + \frac{1}{n_2}}}$
	\end{itemize}
	\item 대립가설의 형태에 따른 기각역과 유의확률
	\begin{center}
		\begin{tabular}{c|c|c}
			대립가설 & 유의수준 $\alpha$에서의 기각역 & 유의확률\\ \hline
			$H_1$: $\mu_1-\mu_2 >\delta_0$ & $t\geq t_{\alpha}(n_1+n_2 - 2)$ & $\pr\left(T\geq t\right)$\\ 
			$H_1$: $\mu_1-\mu_2 <\delta_0$ & $t\leq -t_{\alpha}(n_1+n_2 - 2)$ & $\pr\left(T\leq t\right)$\\ 
			$H_1$: $\mu_1-\mu_2 \neq \delta_0$ & $\left|t\right|\geq t_{\alpha/2}(n_1+n_2 - 2)$ & $\pr\left(\left|T\right|\geq \left|t\right|\right)$
		\end{tabular}
	\end{center}
\end{itemize}
}

\prob{두 약물 A, B에 대하여 효과가 나타나기까지 걸리는 시간을 조사하기 위해 건강 상태가 비슷한 지원자 12명 중에서 6명을 랜덤추출하여 약물 A를, 나머지 6명에게는 약물 B를 주사하여 측정한 결과가 다음과 같았다. 두 약물의 효과가 나타나기까지 걸리는 시간은 분산이 동일한 정규분포를 따른다고 할 때, 약물 효과가 나타나기까지 걸리는 평균 시간에 차이가 있는지 유의수준 5\%에서 검정하고, 유의확률을 구하여라. 또한, 평균 시간의 차이에 대한 95\% 신뢰구간도 구하여라.
\begin{center}
	\begin{tabular}{c|c|c|c|c|c|c}
		\hline
		약물 A & 19.54 & 14.47 & 16.00 & 24.83 & 26.39 & 11.49 \\
		\hline
		약물 B & 15.95 & 25.89 & 20.53 & 15.52 & 14.18 & 16.00 \\
		\hline
	\end{tabular}~ \\~\\\vspace{5mm}
	\begin{tabular}{c|c|c}
		&약물 A&약물 B\\\hline
		표본 크기 & & \\ \hline
		평균 & $~\;\qquad\qquad \;~$&$~\;\qquad\qquad \;~$ \\\hline
		표준편차 & & \\\hline
	\end{tabular}
\end{center}
}\\\\\\
합동 이표본 $t$-검정을 사용할 때에는 \textbf{두 모집단의 분산이 같아야 한다}는 전제 조건에 유의해야 한다. 특히 표본의 크기 $n_1, n_2$가 매우 다를 때에는 전제 조건이 충족되지 않으면 검정의 유효성이 쉽게 상실된다는 것이 알려져 있다.

\pagebreak
~\\현실에서는 분산이 지나치게 큰 상황도 문제가 될 수 있다. 이제 모평균 $\mu$ 와 모표준편차 $\sigma$ 가 모두 미지인 \textbf{정규모집단}에서\footnote{정규모집단이라는 가정에 추론의 타당성이 깊게 의존하고 있다.} 모분산에 대한 유의성검증을 하는 방법을 알아보자.\\
모분산 $\sigma^2$ 의 추정량인 표본분산 $S^2$ 을 이용하여 추정을 하고, 표본분산의 표본분포는 $$\frac{(n-1)S^2}{\sigma^2} \sim \chisq(n-1)$$ 임을 이용하여 모분산에 관한 추정이나 유의성검증을 할 수 있다.\\

\thm{\textbf{모분산의 구간추정} (정규모집단의 경우)\\
모분산 $\sigma^2$ 에 대한 $100(1-\alpha)\%$ 신뢰구간은 다음과 같다. 
$$\left(\frac{(n-1)S^2}{\chisq_{\alpha/2}(n-1)}, \frac{(n-1)S^2}{\chisq_{1-\alpha/2}(n-1)}\right)$$
\textbf{증명}. 연습문제로 남긴다.
}

\thm{\textbf{모분산 $\sigma^2$ 에 관한 추론 \underline{(정규모집단의 경우)}} - $\chisq$ 검정
\begin{itemize}
	\item $H_0$: $\sigma^2 = \sigma_0^2$ 에 대한 검정
	\begin{itemize}
		\item 검정통계량: $\chisq = \ds \frac{(n-1)S^2}{\sigma_0^2} \sim_{H_0} \chisq(n-1)$
		\item 관측값: $\chisq_0 = \ds \frac{(n-1)s^2}{\sigma_0^2}$
	\end{itemize}
	\item 대립가설의 형태에 따른 기각역과 유의확률
	\begin{center}
		\begin{tabular}{c|c|c}
			대립가설 & 유의수준 $\alpha$에서의 기각역 & 유의확률\\ \hline
			$H_1$: $\sigma^2>\sigma_0^2$ & $\chisq_0 \geq \chisq_{\alpha}(n-1)$ & $\pr(\chisq \geq \chisq_0)$\\ 
			$H_1$: $\sigma^2<\sigma_0^2$ & $\chisq_0 \leq \chisq_{1-\alpha}(n-1)$ & $\pr(\chisq \leq \chisq_0)$\\
			$H_1$: $\sigma^2\neq\sigma_0^2$ & $\begin{array}{c}
			\chisq_0 \geq \chisq_{\alpha/2}(n-1) \text{ 또는}\\ \chisq_0 \leq \chisq_{1-\alpha/2}(n-1)
			\end{array}$ & $\begin{array}{c} 2\pr(\chisq \geq \chisq_0) \text{ 또는 } \\ 2\pr(\chisq \leq \chisq_0)
			\end{array}$\\
		\end{tabular}
	\end{center} 
\end{itemize}
양측 검정의 경우 유의확률을 계산하고 2가지 중 1보다 작은 값을 사용한다.
}

\prob{플라스틱 판을 생산하는 한 공장에서 생산되는 판 두께의 표준편차가 1.5mm 를 상회하면 공정에 이상이 있는 것으로 간주한다. 점검을 위해 10개의 판을 랜덤추출하여 두께를 측정한 결과가 mm 단위로 다음과 같이 주어졌다. 과거의 기록에 의하면 판 두께의 분포가 정규분포를 따른다고 할 때, 공정에 이상이 있는지 유의수준 5\%에서 검정하고 유의확률을 구하여라. 또한, 판 두께의 표준편차에 대한 95\% 신뢰구간도 구하여라.
\begin{center}
	226, 228, 226, 225, 232, 228, 227, 229, 225, 230
\end{center}
}~\\\\\\
다음으로는 두 모집단의 표준편차를 비교하는 방법에 대하여 알아보자.\\

\thm{\textbf{모분산의 비 $\sigma_1^2/\sigma_2^2$ 에 관한 추론} (정규모집단의 경우)
\begin{itemize}
	\item 자료구조: $X_1, \dots, X_{n_1}$ 는 $\mc{N}(\mu_1, \sigma_1^2)$ 의 랜덤표본, $Y_1, \dots, Y_{n_2}$ 는 $\mc{N}(\mu_2, \sigma_2^2)$ 의 랜덤표본으로 서로 독립이다.
	\item 추정량: $\widehat{\sigma_1^2/\sigma_2^2} = S_1^2/S_2^2$ \quad ($S_1^2$ 은 $X_i$ 의 표본분산, $S_2^2$ 은 $Y_i$ 의 표본분산)
	\item 표본분포: $\ds \frac{S_1^2/\sigma_1^2}{S_2^2/\sigma_2^2} \sim F(n_1-1, n_2-1)$
	\item $\sigma_1^2/\sigma_2^2$ 의 $100(1-\alpha)\%$ 신뢰구간:
	$$\left(\frac{S_1^2/S_2^2}{F_{\alpha/2}(n_1-1, n_2-1)}, \frac{S_1^2}{S_2^2} \cdot F_{\alpha/2}(n_2-1, n_1-1)\right)$$
\end{itemize}
\textbf{증명}. 신뢰구간의 증명은 연습문제로 남긴다.
}\\

\thm{\textbf{모분산의 비 $\sigma_1^2/\sigma_2^2$ 에 관한 추론 \underline{(정규모집단의 경우)}} - $F$ 검정
	\begin{itemize}
		\item $H_0$: $\sigma_1^2 = \sigma_2^2$ 에 대한 검정
		\begin{itemize}
			\item 검정통계량: $F = \ds \frac{S_1^2}{S_2^2} \sim_{H_0} F(n_1-1, n_2-1)$ 
			\item 관측값: $f = \ds \frac{s_1^2}{s_2^2}$
		\end{itemize}
		\item 대립가설의 형태에 따른 기각역과 유의확률
		\begin{center}
			\begin{tabular}{c|c|c}
				대립가설 & 유의수준 $\alpha$에서의 기각역 & 유의확률\\ \hline
				$H_1$: $\sigma_1^2>\sigma_2^2$ & $f \geq F_{\alpha}(n_1-1, n_2-1)$ & $\pr(F \geq f)$\\ 
				$H_1$: $\sigma_1^2<\sigma_2^2$ & $f \leq 1/F_{\alpha}(n_2-1, n_1-1)$ & $\pr(F \leq f)$\\
				$H_1$: $\sigma_1^2\neq\sigma_2^2$ & $\begin{array}{c}
			 f \geq F_{\alpha/2}(n_1-1, n_2-1)\text{ 또는}\\ f \leq 1/F_{\alpha/2}(n_2-1, n_1-1)
				\end{array}$ & $\begin{array}{c} 2\pr(F \geq f) \text{ 또는 } \\ 2\pr(F \leq f)
				\end{array}$\\
			\end{tabular}
		\end{center} 
	\end{itemize}
	양측 검정의 경우 유의확률을 계산하고 2가지 중 1보다 작은 값을 사용한다.
}

\prob{두 폴리머를 사용할 때의 주입 압축률에 대한 자료이다. 두 폴리머의 압축률의 산포가 다르다는 증거가 있는가를 유의수준 5\%에서 검정하고, 이들의 표준편차의 비에 대한 95\% 신뢰구간을 구하여라.
\begin{center}
	\begin{tabular}{c|cccc}
		Epoxy & 1.75 & 2.12 & 2.05 & 1.97\\ \hline
		MMA Prepolymer & 1.77 & 1.59 & 1.70 & 1.69
	\end{tabular}
\end{center}
}\\\\

\prob{정신분열증환자 9명과 비슷한 연령의 정상인 9명의 해부로부터 뇌세포의 견본도에 대해 특정한 실험을 행하였다. 이 실험에서 각 세포조직에 대하여 1시간 동안에 특정한 효소활동에 의해 형성되는 물질의 양을 측정한 결과가 다음과 같았다. 물음에 답하여라.
\begin{center}
	\begin{tabular}{c|c|c}
		& 평균 & 표준편차 \\ \hline
		정상인 & 39.8 & 8.16 \\ \hline
		환자 & 35.5 & 6.93 \\ \hline
	\end{tabular}
\end{center}
\begin{itemize}
	\item[(1)] 정상인보다 환자의 평균 뇌세포 활동이 저조한지 유의수준 5\%에서 검정하여라.
	\item[(2)] 두 그룹의 사람들의 평균 뇌세포 활동의 차에 대한 95\% 신뢰구간을 구하여라.
	\item[(3)] 정상인의 세포활동의 표준편차가 환자보다 크다고 할 수 있는지 유의수준 5\%에서 검정하여라.
	\item[(4)] 두 모표준편차의 비에 대한 90\% 신뢰구간을 구하여라.
\end{itemize}
}
\pagebreak


