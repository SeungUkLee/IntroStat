\section{표본분포}
모집단 전체를 조사하는 것은 비용이 많이 들고, 또 현실적으로 어렵다. 따라서 통계적 추정을 할 때에는 표본을 뽑아 조사하는 것이 경제적이다.\\

\defn{모집단에서 임의추출한 표본으로부터 얻은 통계량은 확률변수이므로 분포를 가지게 된다. 이를 \textbf{표본분포}(sampling distribution)라 한다.}

\defn{모집단에서 임의추출한 크기 $n$인 표본을 $X_1, \dots, X_n$ 이라 할 때,
\begin{itemize}
	\item \textbf{표본평균}(sample mean): $\overline{X} = \ds \frac{1}{n}\sum_{i=1}^n X_i$
	\item \textbf{표본분산}: $s^2 = \ds \frac{1}{n-1}\sum_{i=1}^n \left(X_i-\overline{X}\right)^2$
\end{itemize}
표본을 뽑을 때마다 표본평균은 달라질 수 있으므로 표본평균 $\overline{X}$는 확률변수가 된다. 따라서, $\ex(\overline{X}), \var(\overline{X}), \sigma(\overline{X})$ 도 계산할 수 있다.
}

\prob{모집단 $\{1, 3, 5, 7\}$ 에서 크기가 2인 표본을 복원추출 할 때, 표본평균 $\overline{X}$의 확률분포가 다음과 같다.
\begin{center}
	\begin{tabular}{c|c|c|c|c|c|c|c|c}
		$\overline{X}$ & 1&2&3&4&5&6&7&합계\\\hline
		$\pr(\overline{X} = \overline{x})$ & $\frac{1}{16}$ & $\frac{1}{8}$ & $a$ & $b$ & $\frac{3}{16}$ & $c$ & $\frac{1}{16}$ &1
		
	\end{tabular}
\end{center}
이 때, $a, b, c$ 의 값을 구하고, 확률변수 $\overline{X}$의 기댓값과 분산을 구하여라.}\\

\defn{$X_i$ 를 $i$-번째로 뽑힌 추출단위의 특성값을 나타내는 확률변수라 하자. 다음 조건을 만족하는 $X_1, \dots, X_n$ 을 \textbf{랜덤표본}(random sample)이라 한다.
\begin{enumerate}
	\item[(1)] (유한모집단) 단순랜덤 비복원추출로 뽑은 표본
	\item[(2)] (무한모집단) $X_i$ 들은 서로 독립이고 각 분포가 모집단 분포와 동일
\end{enumerate}
참고: 유한모집단에서 모집단의 크기가 큰 경우에는 흔히 무한모집단에서의 랜덤표본으로 간주하여 표본분포를 구한 다음 이를 실제표본분포의 근사분포로 사용한다.
}

\thm{모평균이 $\mu$이고 모표준편차가 $\sigma$인 모집단에서\footnote{무한모집단의 경우 복원추출이나 비복원추출이나 큰 차이가 없다.} 복원추출하여 뽑은 크기가 $n$인 표본의 표본평균 $\overline{X}$에 대하여 다음이 성립한다. $$\ex(\overline{X}) = \mu, \: \var(\overline{X}) = \frac{\sigma^2}{n}, \: \sigma(\overline{X}) = \frac{\sigma}{\sqrt{n}}$$
\textbf{증명}. $\overline{X} = \ds \frac{1}{n} \sum_{i=1}^n X_i$ 를 이용한다.\\
$$\ex(\overline{X}) = \ds \frac{1}{n}\ex(X_1+\cdots + X_n) = \frac{n\mu}{n} = \mu$$$$\var(\overline{X}) =\ds \frac{1}{n^2}\var(X_1+\cdots+X_n) = \frac{1}{n^2}\sum_{i=1}^n \var(X_i) = \frac{n\sigma^2}{n^2} = \frac{\sigma^2}{n}$$ \qed\\
\textbf{참고}: 모평균이 $\mu$, 모분산이 $\sigma^2$, 크기가 $N$인 유한모집단에서 크기 $n$인 표본을 비복원추출하는 경우에는 다음이 성립한다. $$\ex(\overline{X}) = \mu, \: \var(\overline{X}) = \frac{N-n}{N-1}\cdot\frac{\sigma^2}{n}$$
여기서 $\sqrt{\frac{N-n}{N-1}}$ 은 finite-population correction factor (FPC) 로 불린다.
}


\prob{연습문제 9.3 에서 구한 $\overline{X}$ 의 기댓값과 분산이 위 정리를 사용하여 구한 것과 일치함을 확인하여라.}

\thm{표본의 크기가 클수록 표본평균과 모평균의 오차가 줄어든다.}

\thm{모집단의 분포가 $\mc{N}(\mu, \sigma^2)$ 일 때, 표본평균 $\overline{X}$는 $\mc{N}(\mu, \sigma^2/n)$ 을 따른다.}

\prob{$\mc{N}(100, 2^2)$ 을 따르는 모집단에서 크기가 4인 표본을 임의추출할 때, 표본평균 $\overline{X}$가 따르는 분포를 구하여라.}

\prob{어느 전기 회사에서 생산하는 전구의 수명을 나타내는 확률변수 $X$에 대하여, $X\sim \mc{N}(2000, 200^2)$ 이라고 한다. 이 회사가 생산한 전구 중 임의추출한 $n$개 전구의 평균 수명을 $\overline{X}$라 할 때, $\pr(1900 \leq \overline{X}\leq 2100)\geq0.9$ 가 성립하기 위한 $n$의 최솟값을 구하여라. 단, $z_{0.05} = 1.65$ 이다.}

\thm{(\textbf{중심극한정리} - Central Limit Theorem) 평균이 $\mu$이고 분산이 $\sigma^2$인 임의의 무한모집단에서 표본의 크기 $n$이 충분히 크면, 랜덤표본의 표본평균 $\overline{X}$는 근사적으로 정규분포 $\mc{N}\left(\mu, \ds \frac{\sigma^2}{n}\right)$ 을 따른다.\footnote{당연히, $n$이 클수록 근사는 정확해진다.}}

\prob{대학 신입생 신장의 평균이 168cm이고 표준편차가 6cm임이 알려져 있다. 100명의 신입생을 단순랜덤추출하는 경우 표본평균이 167cm 이상 169cm 이하일 확률을 구하여라. 단, $z_{0.0475} = 1.67$ 이다.}\\\\\\\\

\pagebreak
